	\subsection{Version mf6.3.0--March 4, 2022}
	
	\underline{NEW FUNCTIONALITY}
	\begin{itemize}
		\item New publications describing MODFLOW~6 were released, including \cite{modflow6api}, \cite{modflow6gwt}, and \cite{modflow6csub}. 
	        \item The GWF-GWF Exchange was updated to support XT3D flow calculations along the edges of connected GWF Models.  The XT3D flow calculation is an alternative to the ghost-node correction and has been shown to provide accurate flow calculations for cell connections that do not meet the control-volume finite-difference requirements.  This new capability allows the GWF-GWF Exchange to tightly couple a wide variety of parent and child grid configurations.  The new XT3D option for the GWF-GWF Exchange can be activated by specifying XT3D in the options block of the GWF-GWF Exchange input file.  The XT3D implementation for GWF-GWF is based on a new generalized coupling method that is described in the Supplemental Technical Information document distributed with this release.
	        \item A new capability was added to support tight coupling of GWT Models through a new GWT-GWT Exchange.  The new GWT-GWT Exchange can be used to connect any two GWT Models in a manner similar to the GWF-GWF Exchange.  This allows transport to be represented from one GWT Model to another.  The GWT-GWT Exchange calculates advective and dispersive fluxes and also supports the Mover Transport (MVT) Package.  This first release of the GWT-GWT Exchange is limited to simulations in which the corresponding GWF Models and GWF-GWF Exchanges are run concurrently within the same simulation; the new capability does not support simulation of transport using groundwater flows saved from a previous simulation; however, this may be supported in the future.  The new GWT-GWT functionality is activated by creating a GWT-GWT input file and specifying a GWT6-GWT6 exchange in the EXCHANGEDATA block of the simulation name file (mfsim.nam).  The GWT-GWT Exchange is based on a new generalized coupling method that is described in the Supplemental Technical Information document distributed with this release.
	        \item This release introduces the Time-Varying hydraulic conductivity (TVK) and the Time-Varying Storage (TVS) options for the Node Property Flow (NPF) and Storage (STO) Packages, respectively, of the GWF Model.  The TVK option is activated by specifying the TVK6 keyword in the OPTIONS block of the NPF Package and by providing a TVK6 input file.  The TVS option is activated by specifying the TVS6 keyword in the OPTIONS block of the STO Package, and by providing a TVS6 input file.  The TVK6 and TVS6 input files are described in input and output guide (mf6io.pdf).  Technical information about the TVK and TVS options is provided in the Supplemental Technical Information report that is provided with the distribution.
	        \item The option already existed to provide binary budget and head file information from a GWF simulation for only the first time step of a given stress period, in which case that information will be used for all time steps in that stress period in a GWT simulation that reads from those files. The behavior of this option has been extended such that if the binary budget and head file information provided for the final stress period of the GWF simulation is for only one time step, that information will be used for any subsequent time steps and stress periods in the GWT simulation. This extended behavior includes as a special case the use of binary budget and head file information from only one time step in only one stress period (for example, a single steady-state GWF stress period) for all time steps in all stress periods in a GWT simulation.
	        \item Added new file-based input capability for the GWT Source and Sink Mixing (SSM) Package.  The SSM Package previously required that all source and sink concentrations be provided as auxiliary variables for corresponding GWF stress packages.  The SSM Package now supports an optional new FILEINPUT block, which can be provided with package names and file names.  Files referenced in the FILEINPUT block can be used to provide time-varying source and sink concentrations.  Descriptions for this new capability are described in the input and output guide (mf6io.pdf) under the GWT SSM Package, and the GWT Stress Package Concentrations (SPC) for list-based input and array-based input.  If used, a single SPC6 input file can be provided for a corresponding GWF stress package.  Either an SPC6 input file or the auxiliary variable approach can be used to supply concentrations for a single GWF stress package, but not both.
	        \item Added BUDGETCSV option to GWF and GWT model output control and to advanced packages (SFR, LAK, MAW, UZF,  MVR, SFT, LKT, MWT, UZT, and MVT) that produce summary tables of budget information.  If activated, this option will cause a comma-separated value (CSV) file of the model budget terms to be written to a user-specified output file.  These output CSV files can be easily read into a spreadsheet or scripting language for plotting of budget terms.
	        \item Added option to use irregular cross sections to define the reach geometry for individual reaches in the SFR Package. Cross-section data for irregular cross-sections are defined in a separate Cross-Section Table Input File. The station-elevation data for an irregular cross section are specified as xfraction and height data, which is converted to station position using the specified reach width (RWID) and elevation using the specified bottom elevation of the reach (RTP). Manning's roughness coefficient fractions can optionally be specified with the xfraction-height data for a irregular cross section to represent roughness coefficient variations in a channel (for example, different channel and overbank Manning's roughness coefficients). SFR Package irregular cross sections and the method used to solve for streamflow in a reach with non-uniform Manning's roughness coefficients is a generalization of the methods used for 8-point cross-sections in the SFR Package for previous versions of MODFLOW \citep{modflowsfr1pack}. 
	\end{itemize}
	
	\underline{EXAMPLES}
	\begin{itemize}
	        \item A new example, ex-gwtgwt-mt3dms-p10, was added to demonstrate the new GWT-GWT Exchange.  
	\end{itemize}

	\textbf{\underline{BUG FIXES AND OTHER CHANGES TO EXISTING FUNCTIONALITY}} \\
	\underline{BASIC FUNCTIONALITY}
	\begin{itemize}
	        \item Fixed a bug that caused the last binary budget and head file information provided by a GWF simulation in a given stress period to be used for all subsequent GWT simulation time steps in that stress period, even if binary budget and head file information was provided for more than one time step in that stress period. In that situation, the GWF time steps must match the GWT time steps one-for-one.
	        \item Boundary packages, such as WEL, DRN and GHB, for example, read lists of data using a general list reader.  For text input, the list of data would not be read correctly if it was longer than 300 characters wide.  Most lists would normally be less than 300 characters unless a large number of auxiliary variables were specified.  In this case, information beyond 300 characters would not be read.  The list reader was modified to use an unlimited character length so that any number of auxiliary variables can be specified.
	        \item The Observation (OBS) functionality can optionally write simulated values to a comma-separated value (CSV) output file.  The OBS input file contains a DIGITS option, which controls the number of digits written to the CSV output file.  In some cases the ``E'' character was not written, making it difficult to read the CSV output file with other programs.  New functionality was programmed to write observations using the maximum number of digits stored internally by the program using the G0 Fortran specifier.  The default value for DIGITS was five, but this has been changed to this maximum number of digits.  This maximum number of digits can also be activated by specifying a zero for DIGITS.  For most applications, observations should be written with the default number of digits so that no precision is lost in the output.
	        \item Added a check to the Flow Model Interface (FMI) Package of the GWT Model that will cause the program to terminate with an error if flow and budget files needed from a flow model cannot be located.
	        \item The Output Control Package did not correctly determine if the end of a stress period was reached if the Adaptive Time Stepping (ATS) option was active.  Output Control works correctly now with ATS.
	        \item Budget information for individual boundary package entries can be written to the model list file in a table form.  The table no longer includes boundaries that are in cells that are dry.
	        \item When the simulated concentration in a cell is negative, which can occur due to the numerical methods used to solve the transport equation, then any sinks present in the cell should not add or remove mass.  The program was modified so that transport through sinks is deactivated if the simulated concentration in a cell is negative.
	        \item Parsing of observation input was corrected so that observation names can have spaces if the observation name is enclosed in quotations.
	        \item The GWF Node Property Flow (NPF) Package can optionally calculate the three components of specific discharge.  The program calculates these components using an interpolation method based on the XT3D method, the simulated flows across each cell face, and distances from the center of the cell to each cell face.  This release contains a fix for a coding error in the calculation of the distance from the center of the cell to the cell face.  The fix will affect the calculated specific discharge for model grids that have variable cell sizes.  Because specific discharge is used in the calculation of the dispersion coefficients for GWT Models, simulated concentrations with this release may be different from simulated concentrations from previous releases; however, these concentration differences are expected to be minor.
	\end{itemize}

	\underline{STRESS PACKAGES}
	\begin{itemize}
	        \item Fixed a bug in the Recharge (RCH) and Evapotranspiration (EVT) Packages that would occur with array-based input (READASARRAYS) combined with the Time-Array Series (TAS) functionality.  Because the TAS functionality only works for DIS and DISV models, this fix does not have any effect on models that use DISU discretization.  The bug would occur when one or more cells in the upper layer were removed using the IDOMAIN capability.  In this special case, the time-array series values did not carry over to the correct locations in the RCH and EVT arrays.  This error would be difficult to identify for complex models, because the model would run to completion without any errors.  The RCH and EVT Packages were modified to maintain direct correspondence between the \texttt{nodelist} and \texttt{bound} arrays with the arrays provided to the TAS utility.  A small and intended consequence of this fix is that the ID2 value written to the binary budget file for array-based RCH and EVT Packages will correspond to the consecutive cell number in the top layer.  
	        \item Recharge and evapotranspiration flows that are written to the GWF Model binary budget file are marked with the text headers ``RCH'' and ``EVT'', respectively.  In order to support the array-based input for SSM concentrations, these text headers are marked as ``RCHA'' and ``EVTA'', respectively, if the READASARRAYS option is specified for the package.  This change will have no effect on simulation results; however, post processing capabilities may need to be adjusted to account for this minor renaming convention.
	        \item When the PRINT\_FLOWS option was used with the Source and Sink Mixing (SSM) Package, the cell ID numbers written to the listing file were incorrect.  The SSM Package was fixed to output the correct cell ID.
	        \item Add new AUTO\_FLOW\_REDUCE\_CSV option for the Well Package.  If activated, then this option will result in information being written to a comma-separated value file for each well and for each time step in which the extraction rate is reduced.  Well extraction rates can be reduced for some groundwater conditions if the AUTO\_FLOW\_REDUCE option is activated.  Information is not written for wells in which the extraction rate is equal to the user-specified extraction rate.
	\end{itemize}

	\underline{ADVANCED STRESS PACKAGES}
	\begin{itemize}
	        \item The Streamflow Routing (SFR) Package would terminate with a floating point error when calculating the stream depth as a function of flow rate, if the flow rate was slightly negative.  Added a conditional check to ensure that the stream depth is calculated as zero if the calculated flow rate is zero or less.
	\end{itemize}

	\underline{SOLUTION}
	\begin{itemize}
	        \item The way in which constant head and constant concentration boundary conditions are handled when the conjugate gradient method is used for the linear solve was modified.  In previous versions, constant head and concentration conditions would result in an asymmetric coefficient matrix.  The program has been modified so that if the conjugate gradient method is selected for the linear solution, then matrix symmetry is preserved by adding adding flows into and out of constant head and concentration cells to the right-hand side of connected cells.
	\end{itemize}


