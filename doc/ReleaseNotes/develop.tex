% Use this template for starting initializing the release notes
% after a release has just been made.
	
	\item \currentmodflowversion

	%\underline{NEW FUNCTIONALITY}
	%\begin{itemize}
	%	\item xxx
	%	\item xxx
	%	\item xxx
	%\end{itemize}

	%\underline{EXAMPLES}
	%\begin{itemize}
	%	\item xxx
	%	\item xxx
	%	\item xxx
	%\end{itemize}

	\textbf{\underline{BUG FIXES AND OTHER CHANGES TO EXISTING FUNCTIONALITY}} \\
	\underline{BASIC FUNCTIONALITY}
	\begin{itemize}
		\item Improve error message if the size of data read from a binary array file is inconsistent with READARRAY control line and variable description keywords.
	%	\item xxx
	%	\item xxx
	\end{itemize}

	%\underline{INTERNAL FLOW PACKAGES}
	%\begin{itemize}
	%	\item xxx
	%	\item xxx
	%	\item xxx
	%\end{itemize}

	%\underline{STRESS PACKAGES}
	%\begin{itemize}
	%	\item xxx
	%	\item xxx
	%	\item xxx
	%\end{itemize}

	\underline{ADVANCED STRESS PACKAGES}
	\begin{itemize}
		\item Added functionality to support zero values for each grid dimension when specifying the CELLID for SFR reaches that are not connected to an underlying groundwater grid cell. For example, for a DIS grid a CELLID of 0 0 0 should be specified for unconnected reaches. Warning messages will be issued if NONE is specified for unconnected reaches. Specifying a CELLID of NONE will eventually be deprecated and will cause MODFLOW 6 to terminate with an error.
	\item Added functionality to support specification of a DNODATA (3.0E+30) BEDLEAK value for LAK package connections to identify lake-GWF connections where conductance is solely a function of aquifer properties in the connected GWF cell and lakebed sediments are assumed to be absent. Warning messages will be issued if NONE is specified for LAK package connections. Specifying a BEDLEAK value equal to NONE will eventually be deprecated and will cause MODFLOW 6 to terminate with an error.
	%	\item xxx
	\end{itemize}

	%\underline{SOLUTION}
	%\begin{itemize}
	%	\item xxx
	%	\item xxx
	%	\item xxx
	%\end{itemize}

	%\underline{EXCHANGES}
	%\begin{itemize}
	%	\item xxx
	%	\item xxx
	%	\item xxx
	%\end{itemize}