% Use this template for starting initializing the release notes
% after a release has just been made.
	
	\item \currentmodflowversion
	
	\underline{NEW FUNCTIONALITY}
	\begin{itemize}
		\item A new Groundwater Energy (GWE) transport model is introduced to the code base for simulating heat transport in the subsurface.  Additional information for activating the GWE model type within a MODFLOW 6 simulation is available within the mf6io.pdf document.  New example problems have been developed for testing and demonstrating GWE capabilities (in addition to other internal tests that help verify the accuracy of GWE); however, additional changes to the code and input may be necessary in response to user needs and further testing.
	%	\item xxx
	%	\item xxx
	\end{itemize}

	%\underline{EXAMPLES}
	%\begin{itemize}
	%	\item xxx
	%	\item xxx
	%	\item xxx
	%\end{itemize}

	%\textbf{\underline{BUG FIXES AND OTHER CHANGES TO EXISTING FUNCTIONALITY}} \\
	%\underline{BASIC FUNCTIONALITY}
	%\begin{itemize}
	%	\item xxx
	%	\item xxx
	%	\item xxx
	%\end{itemize}

	%\underline{INTERNAL FLOW PACKAGES}
	%\begin{itemize}
	%	\item xxx
	%	\item xxx
	%	\item xxx
	%\end{itemize}

	%\underline{STRESS PACKAGES}
	%\begin{itemize}
	%	\item xxx
	%	\item xxx
	%	\item xxx
	%\end{itemize}

	\underline{ADVANCED STRESS PACKAGES}
	\begin{itemize}
		\item Refactoring of the Water Mover package in version 6.4.3 introduced a reduction in performance for GWF models with a large number of movers.  The program was corrected so that performance is similar to previous versions.
	%	\item xxx
	%	\item xxx
	\end{itemize}

	%\underline{SOLUTION}
	%\begin{itemize}
	%	\item xxx
	%	\item xxx
	%	\item xxx
	%\end{itemize}

	%\underline{EXCHANGES}
	%\begin{itemize}
	%	\item xxx
	%	\item xxx
	%	\item xxx
	%\end{itemize}

