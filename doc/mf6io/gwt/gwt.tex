The GWT Model simulates three-dimensional transport of a single solute species in flowing groundwater.  The GWT Model solves the solute transport equation using numerical methods and a generalized control-volume finite-difference approach, which can be used with regular MODFLOW grids (DIS Package) or with unstructured grids (DISV and DISU Packages).  The GWT Model is designed to work with most of the new capabilities released with the GWF Model, including the Newton flow formulation, unstructured grids, advanced packages, and the movement of water between packages.  The GWF and GWT Models operate simultaneously during a \mf simulation to represent coupled groundwater flow and solute transport.  The GWT Model can also run separately from a GWF Model by reading the heads and flows saved by a previously run GWF Model.  The GWT model is also capable of working with the flows from another groundwater flow model, as long as the flows from that model can be written in the correct form to flow and head files.  

The purpose of the GWT Model is to calculate changes in solute concentration in both space and time.  Solute concentrations within an aquifer can change in response to multiple solute transport processes.  These processes include (1) advective transport of solute with flowing groundwater, (2) the combined hydrodynamic dispersion processes of velocity-dependent mechanical dispersion and chemical diffusion, (3) sorption of solutes by the aquifer matrix either by adsorption to individual solid grains or by absorbtion into solid grains, (4) transfer of solute into very low permeability aquifer material (called an immobile domain) where it can be stored and later released, (5) first- or zero-order solute decay or production in response to chemical or biological reactions, (6) mixing with fluids from groundwater sources and sinks, and (7) direct addition of solute mass.

With the present implementation, there can be multiple domains and multiple phases.  There is a single mobile domain, which normally consists of flowing groundwater, and there can be one or more immobile domains.  The GWT Model simulates the dissolved phase of chemical constituents in both the mobile and immobile domains.  The dissolved phase is also referred to in this report as the aqueous phase.  If sorption is represented, then the GWT Model also simulates the solid phase of the chemical constituent in both the mobile and immobile domains.  The dissolved and solid phases of the chemical constituent are tracked in the different domains by the GWT Model and can be reported as output as requested by the user.

This section describes the data files for a \mf Groundwater Transport (GWT) Model.  A GWT Model is added to the simulation by including a GWT entry in the MODELS block of the simulation name file.  There are three types of spatial discretization approaches that can be used with the GWT Model: DIS, DISV, and DISU.  The input instructions for these three packages are not described here in this section on GWT Model input; input instructions for these three packages are described in the section on GWF Model input.

The GWT Model is designed to permit input to be gathered, as it is needed, from many different files.  Likewise, results from the model calculations can be written to a number of output files. The GWT Model Listing File is a key file to which the GWT model output is written.  As \mf runs, information about the GWT Model is written to the GWT Model Listing File, including much of the input data (as a record of the simulation) and calculated results.  Details about the files used by each package are provided in this section on the GWT Model Instructions.

The GWT Model reads a file called the Name File, which specifies most of the files that will be used in a simulation. Several files are always required whereas other files are optional depending on the simulation. The Output Control Package receives instructions from the user to control the amount and frequency of output.  Details about the Name File and the Output Control Package are described in this section.

For the GWT Model, ``flows'' (unless stated otherwise) represent solute mass ``flow'' in mass per time, rather than groundwater flow.  

\subsection{Information for Existing Solute Transport Modelers}
The \mf GWT Model contains most of the functionality of MODFLOW-GWT, MT3DMS, MT3D-USGS and MODFLOW-USG.  The following list summarizes major differences between the GWT Model in \mf and previous MODFLOW-based solute transport programs.

\begin{enumerate}

\item The GWT Model simulates transport of a single chemical species; however, because \mf allows for multiple models of the same type to be included in a single simulation, multiple species can be represented by using multiple GWT Models.

\item There is no specialized flow and transport link file \citep{zheng2001modflow} used to pass the simulated groundwater flows to the transport model.  Instead, simulated flows from the GWF Model are passed in memory to the GWT Model while the program is running.  Alternatively, the GWT Model can read binary flow and head files saved by the GWF Model while it is running.  If the user intends to simulate transport through the advanced stress packages and Water Mover Package, then flows from these advanced packages must also be saved to binary files.  Names for these binary files are provided as input to the FMI Package.

\item The GWT Model is based on a generalized control-volume finite-difference method, which means that solute transport can be simulated using regular MODFLOW grids consisting of layers, rows, and columns, or solute transport can be simulated using unstructured grids.

\item Advection can be simulated using central-in-space weighting, upstream weighting, or an implicit second-order TVD scheme.  The GWT model does not have the Method of Characteristics (particle-based approaches) or an explicit TVD scheme.  Consequently, the GWT Model may require a higher level of spatial discretization than other transport models that use higher order terms for advection dominated systems.  This can be an important limitation for some problems, which require the preservation of sharp solute fronts. 

\item Variable-density flow and transport can be simulated by including a GWF Model and a GWT Model in the same \mf simulation.  The Buoyancy Package should be activated for the GWF Model so that fluid density is calculated as a function of simulated concentration.  If more than one chemical species is represented then the Buoyancy Package allows the simulated concentration for each of them to be used in the density equation of state.   \cite{langevin2020hydraulic} describe the hydraulic-head formation that is implemented in the Buoyancy Package for variable-density groundwater flow and present the results from \mf variable-density simulations.  The variable-density capabilities available in \mf replicate and extend the capabilities available in SEAWAT to include the Newton flow formulation and unstructured grids, for example.  

\item The GWT Model has a Source and Sink Mixing (SSM) Package for representing the effects of GWF stress package inflows and outflows on simulated concentrations.  There are two ways in which users can assign concentrations to the individual features in these stress package.  The first way is to activate a concentration auxiliary variable in the GWF stress package.  In the SSM input file, the user provides the name of the auxiliary variable to be used for concentration.  The second way is to create a special SSMI file, which contains user-assigned time-varying concentrations for stress package boundaries.

\item The GWT model includes the MST and IST Packages.  These two package collectively comprise the capabilities of the MT3DMS Reactions Package.

\item The MST Package contains the linear, Freundlich, and Langmuir isotherms for representing sorption.  The IST Packages contains only the linear isotherm for representation of sorption. 

\item The GWT model was designed so that the user can specify as many immobile domains and necessary to represent observed contaminant transport patterns and solute breakthrough curves.  The effects of an immobile domain are represented using the Immobile Storage and Transfer (IST) Package, and the user can specify as many IST Packages as necessary.  

\item Although there is GWF-GWF Exchange, a GWT-GWT Exchange has not yet been developed to connect multiple transport models, as might be done in a nested grid configuration.  

\item There is no option to automatically run the GWT Model to steady state using a single time step.  This is an option available in MT3DMS \citep{zheng2010supplemental}.  Steady state conditions must be determined by running the transport model under transient conditions until concentrations stabilize.

\item The GWT Model described in this report is capable of simulating solute transport in the advanced stress packages of \mf, including the Lake, Streamflow Routing, Multi-Aquifer Well and Unsaturated Zone Transport Packages.  The present implementation simulates solute advection between package features, such as between two stream reaches, but dispersive transport is not represented.  Likewise, solute transport between the advanced packages and the aquifer occurs only through advection.

\item The GWT Model has not yet been programmed to work with the Skeletal Storage, Compaction, and Subsidence (CSUB) Package for the GWF Model.  

\item There are many other differences between the \mf GWT Model and other solute transport models that work with MODFLOW, especially with regards to program design and input and output.  Descriptions for the GWT input and output are described here.

\end{enumerate}

\subsection{Units of Length and Time}
The GWF Model formulates the groundwater flow equation without using prescribed length and time units. Any consistent units of length and time can be used when specifying the input data for a simulation. This capability gives a certain amount of freedom to the user, but care must be exercised to avoid mixing units.  The program cannot detect the use of inconsistent units.

\subsection{Solute Mass Budget}
A summary of all inflow (sources) and outflow (sinks) of solute mass is called a mass budget.  \mf calculates a mass budget for the overall model as a check on the acceptability of the solution, and to provide a summary of the sources and sinks of mass to the flow system.  The solute mass budget is printed to the GWT Model Listing File for selected time steps.

\subsection{Time Stepping}

For the present implementation of the GWT Model, all terms in the solute transport equation are solved implicitly.  With the implicit approach applied to the transport equation, it is possible to take relatively large time steps and efficiently obtain a stable solution.  If the time steps are too large, however, accuracy of the model results will suffer, so there is usually some compromise required between the desired level of accuracy and length of the time step.  An assessment of accuracy can be performed by simply running simulations with shorter time steps and comparing results.

In \mf time step lengths are controlled by the user and specified in the Temporal Discretization (TDIS) input file.  When the flow model and transport model are included in the same simulation, then the length of the time step specified in TDIS is used for both models.  If the GWT Model runs in a separate simulation from the GWT Model, then the time steps used for the transport model can be different, and likely shorter, than the time steps used for the flow solution.  Instructions for specifying time steps are described in the TDIS section of this user guide; additional information on GWF and GWT configurations are in the Flow Model Interface section.  



\newpage
\subsection{GWT Model Name File}
The PRT Model Name File specifies the options and packages that are active for a PRT model.  The Name File contains two blocks: OPTIONS  and PACKAGES. The length of each line must be 299 characters or less. The lines in each block can be in any order.  Files listed in the PACKAGES block must exist when the program starts. 

Comment lines are indicated when the first character in a line is one of the valid comment characters.  Commented lines can be located anywhere in the file. Any text characters can follow the comment character. Comment lines have no effect on the simulation; their purpose is to allow users to provide documentation about a particular simulation. 

\vspace{5mm}
\subsubsection{Structure of Blocks}
\lstinputlisting[style=blockdefinition]{./mf6ivar/tex/prt-nam-options.dat}
\lstinputlisting[style=blockdefinition]{./mf6ivar/tex/prt-nam-packages.dat}

\vspace{5mm}
\subsubsection{Explanation of Variables}
\begin{description}
% DO NOT MODIFY THIS FILE DIRECTLY.  IT IS CREATED BY mf6ivar.py 

\item \textbf{Block: OPTIONS}

\begin{description}
\item \texttt{list}---is name of the listing file to create for this PRT model.  If not specified, then the name of the list file will be the basename of the PRT model name file and the '.lst' extension.  For example, if the PRT name file is called ``my.model.nam'' then the list file will be called ``my.model.lst''.

\item \texttt{PRINT\_INPUT}---keyword to indicate that the list of all model stress package information will be written to the listing file immediately after it is read.

\item \texttt{PRINT\_FLOWS}---keyword to indicate that the list of all model package flow rates will be printed to the listing file for every stress period time step in which ``BUDGET PRINT'' is specified in Output Control.  If there is no Output Control option and ``PRINT\_FLOWS'' is specified, then flow rates are printed for the last time step of each stress period.

\item \texttt{SAVE\_FLOWS}---keyword to indicate that all model package flow terms will be written to the file specified with ``BUDGET FILEOUT'' in Output Control.

\end{description}
\item \textbf{Block: PACKAGES}

\begin{description}
\item \texttt{ftype}---is the file type, which must be one of the following character values shown in table~\ref{table:ftype-prt}. Ftype may be entered in any combination of uppercase and lowercase.

\item \texttt{fname}---is the name of the file containing the package input.  The path to the file should be included if the file is not located in the folder where the program was run.

\item \texttt{pname}---is the user-defined name for the package. PNAME is restricted to 16 characters.  No spaces are allowed in PNAME.  PNAME character values are read and stored by the program for stress packages only.  These names may be useful for labeling purposes when multiple stress packages of the same type are located within a single PRT Model.  If PNAME is specified for a stress package, then PNAME will be used in the flow budget table in the listing file; it will also be used for the text entry in the cell-by-cell budget file.  PNAME is case insensitive and is stored in all upper case letters.

\end{description}


\end{description}

\begin{table}[H]
\caption{Ftype values described in this report.  The \texttt{Pname} column indicates whether or not a package name can be provided in the name file.  The capability to provide a package name also indicates that the PRT Model can have more than one package of that Ftype}
\small
\begin{center}
\begin{tabular*}{\columnwidth}{l l l}
\hline
\hline
Ftype & Input File Description & \texttt{Pname}\\
\hline
DIS6 & Rectilinear Discretization Input File \\
DISV6 & Discretization by Vertices Input File \\
MIP6 & Model Input File \\
FMI6 & Flow Model Interface Package &  \\ 
PRP6 & Particle Release Point Package \\
OC6 & Output Control Option \\
OBS6 & Observations Option \\
\hline 
\end{tabular*}
\label{table:ftypeprt}
\end{center}
\normalsize
\end{table}

\vspace{5mm}
\subsubsection{Example Input File}
\lstinputlisting[style=inputfile]{./mf6ivar/examples/prt-nam-example.dat}



%\newpage
%\subsection{Structured Discretization (DIS) Input File}
%\input{gwf/dis}

%\newpage
%\subsection{Discretization with Vertices (DISV) Input File}
%\input{gwf/disv}

%\newpage
%\subsection{Unstructured Discretization (DISU) Input File}
%\input{gwf/disu}

\newpage
\subsection{Initial Conditions (IC) Package}
Initial Conditions (IC) Package information is read from the file that is specified by ``IC6'' as the file type.  Only one IC Package can be specified for a GWE model. 

\vspace{5mm}
\subsubsection{Structure of Blocks}
%\lstinputlisting[style=blockdefinition]{./mf6ivar/tex/gwe-ic-options.dat}
\lstinputlisting[style=blockdefinition]{./mf6ivar/tex/gwe-ic-griddata.dat}

\vspace{5mm}
\subsubsection{Explanation of Variables}
\begin{description}
% DO NOT MODIFY THIS FILE DIRECTLY.  IT IS CREATED BY mf6ivar.py 

\item \textbf{Block: OPTIONS}

\begin{description}
\item \texttt{EXPORT\_ARRAY\_ASCII}---keyword that specifies input griddata arrays should be written to layered ascii output files.

\end{description}
\item \textbf{Block: GRIDDATA}

\begin{description}
\item \texttt{strt}---is the initial (starting) temperature---that is, the temperature at the beginning of the GWE Model simulation.  STRT must be specified for all GWE Model simulations. One value is read for every model cell.

\end{description}


\end{description}

\vspace{5mm}
\subsubsection{Example Input File}
\lstinputlisting[style=inputfile]{./mf6ivar/examples/gwe-ic-example.dat}



\newpage
\subsection{Output Control (OC) Option}
Input to the Output Control Option of the Particle Tracking Model is read from the file that is specified as type ``OC6'' in the Name File. If no ``OC6'' file is specified, default output control is used. The Output Control Option determines how and when particle mass budgets are printed to the listing file and/or written to a separate binary output file.  Under the default settings, the particle mass budget is written to the Listing File at the end of every stress period.  The particle mass budget is also written to the list file if the simulation terminates prematurely due to failed convergence.

Output Control data must be specified using words.  The numeric codes supported in earlier MODFLOW versions can no longer be used.

For the PRINT and SAVE options, there is no option to specify individual layers.  Whenever the budget array is printed or saved, all layers are printed or saved.

\vspace{5mm}
\subsubsection{Structure of Blocks}
\vspace{5mm}

\noindent \textit{FOR EACH SIMULATION}
\lstinputlisting[style=blockdefinition]{./mf6ivar/tex/prt-oc-options.dat}
\vspace{5mm}
\noindent \textit{FOR ANY STRESS PERIOD}
\lstinputlisting[style=blockdefinition]{./mf6ivar/tex/prt-oc-period.dat}

\vspace{5mm}
\subsubsection{Explanation of Variables}
\begin{description}
% DO NOT MODIFY THIS FILE DIRECTLY.  IT IS CREATED BY mf6ivar.py 

\item \textbf{Block: OPTIONS}

\begin{description}
\item \texttt{BUDGET}---keyword to specify that record corresponds to the budget.

\item \texttt{FILEOUT}---keyword to specify that an output filename is expected next.

\item \texttt{budgetfile}---name of the output file to write budget information.

\item \texttt{BUDGETCSV}---keyword to specify that record corresponds to the budget CSV.

\item \texttt{budgetcsvfile}---name of the comma-separated value (CSV) output file to write budget summary information.  A budget summary record will be written to this file for each time step of the simulation.

\item \texttt{TRACK}---keyword to specify that record corresponds to a binary track file.

\item \texttt{trackfile}---name of the output file to write tracking information.

\item \texttt{TRACKCSV}---keyword to specify that record corresponds to a CSV track file.

\item \texttt{trackcsvfile}---name of the comma-separated value (CSV) file to write tracking information.

\item \texttt{TRACK\_RELEASE}---keyword to indicate that particle tracking output is to be written when a particle is released

\item \texttt{TRACK\_EXIT}---keyword to indicate that particle tracking output is to be written when a particle exits a cell

\item \texttt{TRACK\_TIMESTEP}---keyword to indicate that particle tracking output is to be written at the end of each time step

\item \texttt{TRACK\_TERMINATE}---keyword to indicate that particle tracking output is to be written when a particle terminates for any reason

\item \texttt{TRACK\_WEAKSINK}---keyword to indicate that particle tracking output is to be written when a particle exits a weak sink (a cell which removes some but not all inflow from adjacent cells)

\item \texttt{TRACK\_USERTIME}---keyword to indicate that particle tracking output is to be written at user-specified times, provided as double precision values to the TRACK\_TIMES or TRACK\_TIMESFILE options

\item \texttt{TRACK\_TIMES}---keyword indicating tracking times will follow

\item \texttt{times}---times to track, relative to the beginning of the simulation.

\item \texttt{TRACK\_TIMESFILE}---keyword indicating tracking times file name will follow

\item \texttt{timesfile}---name of the tracking times file

\end{description}
\item \textbf{Block: PERIOD}

\begin{description}
\item \texttt{iper}---integer value specifying the starting stress period number for which the data specified in the PERIOD block apply.  IPER must be less than or equal to NPER in the TDIS Package and greater than zero.  The IPER value assigned to a stress period block must be greater than the IPER value assigned for the previous PERIOD block.  The information specified in the PERIOD block will continue to apply for all subsequent stress periods, unless the program encounters another PERIOD block.

\item \texttt{SAVE}---keyword to indicate that information will be saved this stress period.

\item \texttt{PRINT}---keyword to indicate that information will be printed this stress period.

\item \texttt{rtype}---type of information to save or print.  Can only be BUDGET.

\item \texttt{ocsetting}---specifies the steps for which the data will be saved.

\begin{lstlisting}[style=blockdefinition]
ALL
FIRST
LAST
FREQUENCY <frequency>
STEPS <steps(<nstp)>
\end{lstlisting}

\item \texttt{ALL}---keyword to indicate save for all time steps in period.

\item \texttt{FIRST}---keyword to indicate save for first step in period. This keyword may be used in conjunction with other keywords to print or save results for multiple time steps.

\item \texttt{LAST}---keyword to indicate save for last step in period. This keyword may be used in conjunction with other keywords to print or save results for multiple time steps.

\item \texttt{frequency}---save at the specified time step frequency. This keyword may be used in conjunction with other keywords to print or save results for multiple time steps.

\item \texttt{steps}---save for each step specified in STEPS. This keyword may be used in conjunction with other keywords to print or save results for multiple time steps.

\end{description}


\end{description}

\vspace{5mm}
\subsubsection{Example Input File}
\lstinputlisting[style=inputfile]{./mf6ivar/examples/prt-oc-example.dat}


\newpage
\subsection{Observation (OBS) Utility for a GWT Model}
\input{gwt/gwt-obs}

\newpage
\subsection{Advection (ADV) Package}
Advection (ADV) Package information is read from the file that is specified by ``ADV6'' as the file type.  Only one ADV Package can be specified for a GWE model. 

\vspace{5mm}
\subsubsection{Structure of Blocks}
\lstinputlisting[style=blockdefinition]{./mf6ivar/tex/gwe-adv-options.dat}

\vspace{5mm}
\subsubsection{Explanation of Variables}
\begin{description}
% DO NOT MODIFY THIS FILE DIRECTLY.  IT IS CREATED BY mf6ivar.py 

\item \textbf{Block: OPTIONS}

\begin{description}
\item \texttt{scheme}---scheme used to solve the advection term.  Can be upstream, central, or TVD.  If not specified, upstream weighting is the default weighting scheme.

\end{description}


\end{description}

\vspace{5mm}
\subsubsection{Example Input File}
\lstinputlisting[style=inputfile]{./mf6ivar/examples/gwe-adv-example.dat}



\newpage
\subsection{Dispersion (DSP) Package}
Dispersion (DSP) Package information is read from the file that is specified by ``DSP6'' as the file type.  Only one DSP Package can be specified for a GWT model.  By default, the DSP Package uses the mathematical formulation presented for the XT3D option of the NPF Package to represent full three-dimensional anisotropy in groundwater flow.  XT3D can be computationally expensive and can be turned off to use a simplified and approximate form of the dispersion equations.  For most problems, however, XT3D will be required to accurately represent dispersion.

\vspace{5mm}
\subsubsection{Structure of Blocks}
\lstinputlisting[style=blockdefinition]{./mf6ivar/tex/gwt-dsp-options.dat}
\lstinputlisting[style=blockdefinition]{./mf6ivar/tex/gwt-dsp-griddata.dat}

\vspace{5mm}
\subsubsection{Explanation of Variables}
\begin{description}
% DO NOT MODIFY THIS FILE DIRECTLY.  IT IS CREATED BY mf6ivar.py 

\item \textbf{Block: OPTIONS}

\begin{description}
\item \texttt{XT3D\_OFF}---deactivate the xt3d method and use the faster and less accurate approximation.  This option may provide a fast and accurate solution under some circumstances, such as when flow aligns with the model grid, there is no mechanical dispersion, or when the longitudinal and transverse dispersivities are equal.  This option may also be used to assess the computational demand of the XT3D approach by noting the run time differences with and without this option on.

\item \texttt{XT3D\_RHS}---add xt3d terms to right-hand side, when possible.  This option uses less memory, but may require more iterations.

\item \texttt{EXPORT\_ARRAY\_ASCII}---keyword that specifies input griddata arrays should be written to layered ascii output files.

\end{description}
\item \textbf{Block: GRIDDATA}

\begin{description}
\item \texttt{diffc}---effective molecular diffusion coefficient.

\item \texttt{alh}---longitudinal dispersivity in horizontal direction.  If flow is strictly horizontal, then this is the longitudinal dispersivity that will be used.  If flow is not strictly horizontal or strictly vertical, then the longitudinal dispersivity is a function of both ALH and ALV.  If mechanical dispersion is represented (by specifying any dispersivity values) then this array is required.

\item \texttt{alv}---longitudinal dispersivity in vertical direction.  If flow is strictly vertical, then this is the longitudinal dispsersivity value that will be used.  If flow is not strictly horizontal or strictly vertical, then the longitudinal dispersivity is a function of both ALH and ALV.  If this value is not specified and mechanical dispersion is represented, then this array is set equal to ALH.

\item \texttt{ath1}---transverse dispersivity in horizontal direction.  This is the transverse dispersivity value for the second ellipsoid axis.  If flow is strictly horizontal and directed in the x direction (along a row for a regular grid), then this value controls spreading in the y direction.  If mechanical dispersion is represented (by specifying any dispersivity values) then this array is required.

\item \texttt{ath2}---transverse dispersivity in horizontal direction.  This is the transverse dispersivity value for the third ellipsoid axis.  If flow is strictly horizontal and directed in the x direction (along a row for a regular grid), then this value controls spreading in the z direction.  If this value is not specified and mechanical dispersion is represented, then this array is set equal to ATH1.

\item \texttt{atv}---transverse dispersivity when flow is in vertical direction.  If flow is strictly vertical and directed in the z direction, then this value controls spreading in the x and y directions.  If this value is not specified and mechanical dispersion is represented, then this array is set equal to ATH2.

\end{description}


\end{description}

\vspace{5mm}
\subsubsection{Example Input File}
\lstinputlisting[style=inputfile]{./mf6ivar/examples/gwt-dsp-example.dat}



\newpage
\subsection{Source and Sink Mixing (SSM) Package}
Source and Sink Mixing (SSM) Package information is read from the file that is specified by ``SSM6'' as the file type.  Only one SSM Package can be specified for a GWT model.  The SSM Package is required if the flow model has any stress packages.

The SSM Package is used to add or remove solute mass from GWT model cells based on inflows and outflows from GWF stress packages.  If a GWF stress package provides flow into a model cell, that flow can be assigned a user-specified concentration.  This user-specified concentration must be entered as an auxiliary variable in the flow package.  In the SOURCES block below, the user provides the name of the package and the name of the auxiliary variable containing concentration values for each boundary.  As described below for srctype, there are multiple options for defining this behavior.  If a flow package specified here is also represented using an advanced transport package (SFT, LKT, MWT, or UZT), then the advanced transport package will override SSM calculations for that package.

If the user does not enter a record for a GWF stress package in the SOURCES block, then inflow to the GWT model is assigned a concentration value of zero.  For negative flow rates in GWF stress packages (sinks), the sink concentration is set to the calculated cell concentration.

\vspace{5mm}
\subsubsection{Structure of Blocks}
\lstinputlisting[style=blockdefinition]{./mf6ivar/tex/gwt-ssm-options.dat}
\lstinputlisting[style=blockdefinition]{./mf6ivar/tex/gwt-ssm-sources.dat}

\vspace{5mm}
\subsubsection{Explanation of Variables}
\begin{description}
% DO NOT MODIFY THIS FILE DIRECTLY.  IT IS CREATED BY mf6ivar.py 

\item \textbf{Block: OPTIONS}

\begin{description}
\item \texttt{PRINT\_FLOWS}---keyword to indicate that the list of SSM flow rates will be printed to the listing file for every stress period time step in which ``BUDGET PRINT'' is specified in Output Control.  If there is no Output Control option and ``PRINT\_FLOWS'' is specified, then flow rates are printed for the last time step of each stress period.

\item \texttt{SAVE\_FLOWS}---keyword to indicate that SSM flow terms will be written to the file specified with ``BUDGET FILEOUT'' in Output Control.

\end{description}
\item \textbf{Block: SOURCES}

\begin{description}
\item \texttt{pname}---name of the flow package for which an auxiliary variable contains a source concentration.  If this flow package is represented using an advanced transport package (SFT, LKT, MWT, or UZT), then the advanced transport package will override SSM terms specified here.

\item \texttt{srctype}---keyword indicating how concentration will be assigned for sources and sinks.  Keyword must be specified as either AUX or AUXMIXED.  For both options the user must provide an auxiliary variable in the corresponding flow package.  The auxiliary variable must have the same name as the AUXNAME value that follows.  If the AUX keyword is specified, then the auxiliary variable specified by the user will be assigned as the concentration value for groundwater sources (flows with a positive sign).  For negative flow rates (sinks), groundwater will be withdrawn from the cell at the simulated concentration of the cell.  The AUXMIXED option provides an alternative method for how to determine the concentration of sinks.  If the cell concentration is larger than the user-specified auxiliary concentration, then the concentration of groundwater withdrawn from the cell will be assigned as the user-specified concentration.  Alternatively, if the user-specified auxiliary concentration is larger than the cell concentration, then groundwater will be withdrawn at the cell concentration.  Thus, the AUXMIXED option is designed to work with the Evapotranspiration (EVT) and Recharge (RCH) Packages where water may be withdrawn at a concentration that is less than the cell concentration.

\item \texttt{auxname}---name of the auxiliary variable in the package PNAME.  This auxiliary variable must exist and be specified by the user in that package.  The values in this auxiliary variable will be used to set the concentration associated with the flows for that boundary package.

\end{description}
\item \textbf{Block: FILEINPUT}

\begin{description}
\item \texttt{pname}---name of the flow package for which an SPC6 input file contains a source concentration.  If this flow package is represented using an advanced transport package (SFT, LKT, MWT, or UZT), then the advanced transport package will override SSM terms specified here.

\item \texttt{SPC6}---keyword to specify that record corresponds to a source sink mixing input file.

\item \texttt{FILEIN}---keyword to specify that an input filename is expected next.

\item \texttt{spc6\_filename}---character string that defines the path and filename for the file containing source and sink input data for the flow package. The SPC6\_FILENAME file is a flexible input file that allows concentrations to be specified by stress period and with time series. Instructions for creating the SPC6\_FILENAME input file are provided in the next section on file input for boundary concentrations.

\item \texttt{MIXED}---keyword to specify that these stress package boundaries will have the mixed condition.  The MIXED condition is described in the SOURCES block for AUXMIXED.  The MIXED condition allows for water to be withdrawn at a concentration that is less than the cell concentration.  It is intended primarily for representing evapotranspiration.

\end{description}


\end{description}

\vspace{5mm}
\subsubsection{Example Input File}
\lstinputlisting[style=inputfile]{./mf6ivar/examples/gwt-ssm-example.dat}



\newpage
\subsection{Mobile Storage and Transfer (MST) Package}
\input{gwt/mst}

\newpage
\subsection{Immobile Storage and Transfer (IST) Package}
Immobile Storage and Transfer (IST) Package information is read from the file that is specified by ``IST6'' as the file type.  Any number of IST Packages can be specified for a single GWT model.  This allows the user to specify triple porosity systems, or systems with as many immobile domains as necessary. 

Subsequent to MODFLOW Version 6.4.1, substantial changes were made to the input parameter definitions and conceptualization of the IST Package.  These changes are described in Chapter 9 of the MODFLOW 6 Supplemental Technical Information document that is included with the distribution.

\vspace{5mm}
\subsubsection{Structure of Blocks}
\lstinputlisting[style=blockdefinition]{./mf6ivar/tex/gwt-ist-options.dat}
\lstinputlisting[style=blockdefinition]{./mf6ivar/tex/gwt-ist-griddata.dat}

\vspace{5mm}
\subsubsection{Explanation of Variables}
\begin{description}
% DO NOT MODIFY THIS FILE DIRECTLY.  IT IS CREATED BY mf6ivar.py 

\item \textbf{Block: OPTIONS}

\begin{description}
\item \texttt{SAVE\_FLOWS}---keyword to indicate that IST flow terms will be written to the file specified with ``BUDGET FILEOUT'' in Output Control.

\item \texttt{BUDGET}---keyword to specify that record corresponds to the budget.

\item \texttt{FILEOUT}---keyword to specify that an output filename is expected next.

\item \texttt{budgetfile}---name of the binary output file to write budget information.

\item \texttt{BUDGETCSV}---keyword to specify that record corresponds to the budget CSV.

\item \texttt{budgetcsvfile}---name of the comma-separated value (CSV) output file to write budget summary information.  A budget summary record will be written to this file for each time step of the simulation.

\item \texttt{SORPTION}---is a text keyword to indicate that sorption will be activated.  Use of this keyword requires that BULK\_DENSITY and DISTCOEF are specified in the GRIDDATA block.  The linear sorption isotherm is the only isotherm presently supported in the IST Package.

\item \texttt{FIRST\_ORDER\_DECAY}---is a text keyword to indicate that first-order decay will occur.  Use of this keyword requires that DECAY and DECAY\_SORBED (if sorption is active) are specified in the GRIDDATA block.

\item \texttt{ZERO\_ORDER\_DECAY}---is a text keyword to indicate that zero-order decay will occur.  Use of this keyword requires that DECAY and DECAY\_SORBED (if sorption is active) are specified in the GRIDDATA block.

\item \texttt{CIM}---keyword to specify that record corresponds to immobile concentration.

\item \texttt{cimfile}---name of the output file to write immobile concentrations.  This file is a binary file that has the same format and structure as a binary head and concentration file.  The value for the text variable written to the file is CIM.  Immobile domain concentrations will be written to this file at the same interval as mobile domain concentrations are saved, as specified in the GWT Model Output Control file.

\item \texttt{PRINT\_FORMAT}---keyword to specify format for printing to the listing file.

\item \texttt{columns}---number of columns for writing data.

\item \texttt{width}---width for writing each number.

\item \texttt{digits}---number of digits to use for writing a number.

\item \texttt{format}---write format can be EXPONENTIAL, FIXED, GENERAL, or SCIENTIFIC.

\end{description}
\item \textbf{Block: GRIDDATA}

\begin{description}
\item \texttt{porosity}---porosity of the immobile domain specified as the immobile domain pore volume per immobile domain volume.

\item \texttt{volfrac}---fraction of the cell volume that consists of this immobile domain.  The sum of all immobile domain volume fractions must be less than one.

\item \texttt{zetaim}---mass transfer rate coefficient between the mobile and immobile domains, in dimensions of per time.

\item \texttt{cim}---initial concentration of the immobile domain in mass per length cubed.  If CIM is not specified, then it is assumed to be zero.

\item \texttt{decay}---is the rate coefficient for first or zero-order decay for the aqueous phase of the immobile domain.  A negative value indicates solute production.  The dimensions of decay for first-order decay is one over time.  The dimensions of decay for zero-order decay is mass per length cubed per time.  Decay will have no effect on simulation results unless either first- or zero-order decay is specified in the options block.

\item \texttt{decay\_sorbed}---is the rate coefficient for first or zero-order decay for the sorbed phase of the immobile domain.  A negative value indicates solute production.  The dimensions of decay\_sorbed for first-order decay is one over time.  The dimensions of decay\_sorbed for zero-order decay is mass of solute per mass of aquifer per time.  If decay\_sorbed is not specified and both decay and sorption are active, then the program will terminate with an error.  decay\_sorbed will have no effect on simulation results unless the SORPTION keyword and either first- or zero-order decay are specified in the options block.

\item \texttt{bulk\_density}---is the bulk density of this immobile domain in mass per length cubed.  Bulk density is defined as the immobile domain solid mass per volume of the immobile domain.  bulk\_density is not required unless the SORPTION keyword is specified in the options block.  If the SORPTION keyword is not specified in the options block, bulk\_density will have no effect on simulation results.

\item \texttt{distcoef}---is the distribution coefficient for the equilibrium-controlled linear sorption isotherm in dimensions of length cubed per mass.  distcoef is not required unless the SORPTION keyword is specified in the options block.  If the SORPTION keyword is not specified in the options block, distcoef will have no effect on simulation results.

\end{description}


\end{description}

\vspace{5mm}
\subsubsection{Example Input File}
\lstinputlisting[style=inputfile]{./mf6ivar/examples/gwt-ist-example.dat}



\newpage
\subsection{Constant Concentration (CNC) Package}
\input{gwt/cnc}

\newpage
\subsection{Mass Source Loading (SRC) Package}
\input{gwt/src}

\newpage
\subsection{Streamflow Transport (SFT) Package}
\input{gwt/sft}

\newpage
\subsection{Lake Transport (LKT) Package}
\input{gwt/lkt}

\newpage
\subsection{Multi-Aquifer Well Transport (MWT) Package}
\input{gwt/mwt}

\newpage
\subsection{Unsaturated Zone Transport (UZT) Package}
\input{gwt/uzt}

\newpage
\subsection{Flow Model Interface (FMI) Package}
Flow Model Interface (FMI) Package information is read from the file that is specified by ``FMI6'' as the file type.  The FMI Package is required, and only one FMI Package can be specified for a PRT model.

For most simulations, the PRT Model needs groundwater flows for every cell in the model grid, for all boundary conditions, and for other terms, such as the flow of water in or out of storage.  The FMI Package is the interface between the PRT Model and simulated groundwater flows provided by a corresponding GWF Model that is running concurrently within the simulation or from binary budget files that were created from a previous GWF model run.  The following are several different FMI simulation cases:

\begin{itemize}

\item Flows are provided by a corresponding GWF Model running in the same simulation---in this case, all groundwater flows are calculated by the corresponding GWF Model and provided through FMI to the transport model.  This is a common use case in which the user wants to run the flow and particle-tracking models as part of a single simulation.  The GWF and PRT models must be part of a GWF-PRT Exchange that is listed in mfsim.nam.  If a GWF-PRT Exchange is specified by the user, then the user does not need to specify an FMI Package input file for the simulation, unless an FMI option is needed.  If a GWF-PRT Exchange is specified and the FMI Package is specified, then the PACKAGEDATA block below is not read or used.

\item Flows are provided from a previous GWF model simulation---in this case FMI should be provided in the PRT name file and the head and budget files should be listed in the FMI PACKAGEDATA block.  In this case, FMI reads the simulated head and flows from these files and makes them available to the particle-trcking model.  There are some additional considerations when the heads and flows are provided from binary files.

\begin{itemize}
\item The binary budget file must contain the simulated flows for all of the packages that were included in the GWF model run.  Saving of flows can be activated for all packages by specifying ``SAVE\_FLOWS'' as an option in the GWF name file.  The GWF Output Control Package must also have ``SAVE BUDGET ALL'' specified.  The easiest way to ensure that all flows and heads are saved is to use the following simple form of a GWF Output Control file:

\begin{verbatim}
BEGIN OPTIONS
  HEAD FILEOUT mymodel.hds
  BUDGET FILEOUT mymodel.bud
END OPTIONS

BEGIN PERIOD 1
  SAVE HEAD ALL
  SAVE BUDGET ALL
END PERIOD
\end{verbatim}

\item The binary budget file must have the same number of budget terms listed for each time step.  This will always be the case when the binary budget file is created by \mf.
\item The binary heads file must have heads saved for all layers in the model.  This will always be the case when the binary head file is created by \mf.  This was not always the case as previous MODFLOW versions allowed different save options for each layer.
\item If the binary budget and head files have more than one time step for a single stress period, then the budget and head information must be contained within the binary file for every time step in the simulation stress period.
\item The binary budget and head files must correspond in terms of information stored for each time step and stress period.
\item If the binary budget and head files have information provided for only the first time step of a given stress period, this information will be used for all time steps in that stress period in the PRT simulation. If the final (or only) stress period in the binary budget and head files contains data for only one time step, this information will be used for any subsequent time steps and stress periods in the PRT simulation. This makes it possible to provide flows, for example, from a steady-state GWF stress period and have those flows used for all PRT time steps in that stress period, for all remaining time steps in the PRT simulation, or for all time steps throughout the entire GWT simulation. With this option, it is possible to have smaller time steps in the PRT simulation than the time steps used in the GWF simulation. Note that this cannot be done when the GWF and PRT models are run in the same simulation, because in that case, both models are solved over the same sequence of time steps and stress periods, as listed in the TDIS Package. The option to read flows from a previous GWF simulation via Flow Model Interface may offer an efficient alternative to running both models in the same simulation, but comes at the cost of having potentially very large budget files.
\end{itemize}

\end{itemize}

\noindent Determination of which FMI use case to invoke requires careful consideration of the different advantages and disadvantages of each case.  For example, running PRT and GWF in the same simulation can often be faster because GWF flows are passed through memory to the PRT model instead of being written to files.  The disadvantage of this approach is that the same time step lengths must be used for both GWF and PRT.  Ultimately, it should be relatively straightforward to test different ways in which GWF and PRT interact and select the use case most appropriate for the particular problem. 

\vspace{5mm}
\subsubsection{Structure of Blocks}
\lstinputlisting[style=blockdefinition]{./mf6ivar/tex/prt-fmi-packagedata.dat}

\vspace{5mm}
\subsubsection{Explanation of Variables}
\begin{description}
% DO NOT MODIFY THIS FILE DIRECTLY.  IT IS CREATED BY mf6ivar.py 

\item \textbf{Block: OPTIONS}

\begin{description}
\item \texttt{SAVE\_FLOWS}---keyword to indicate that FMI flow terms will be written to the file specified with ``BUDGET FILEOUT'' in Output Control.

\end{description}
\item \textbf{Block: PACKAGEDATA}

\begin{description}
\item \texttt{flowtype}---is the word GWFBUDGET or GWFHEAD.  If GWFBUDGET is specified, then the corresponding file must be a budget file from a previous GWF Model run.

\item \texttt{FILEIN}---keyword to specify that an input filename is expected next.

\item \texttt{fname}---is the name of the file containing flows.  The path to the file should be included if the file is not located in the folder where the program was run.

\end{description}


\end{description}

\vspace{5mm}
\subsubsection{Example Input File}
\lstinputlisting[style=inputfile]{./mf6ivar/examples/prt-fmi-example.dat}



\newpage
\subsection{Mover Transport (MVT) Package}
\input{gwt/mvt}

