For consistency with earlier versions of MODFLOW (specifically, MODFLOW-2000 and MODFLOW-2005), \programname{} supports an ``Observation'' utility. Unlike the earlier versions of MODFLOW, the Observation utility of \programname{} does not require input of ``observed'' values, which typically were field- or lab-measured values. The Observation utility described here provides options for extracting numeric values of interest generated in the course of a model run. The Observation utility does not calculate residual values (differences between observed and model-calculated values). Output generated by the Observation utility is designed to facilitate further processing. For convenience and for consistency with earlier terminology, individual entries of the Observation utility are referred to as ``observations.''

Input for the Observation utility is read from one or more input files, where each file is associated with a specific model or package. For extracting values simulated by a GWF model, input is read from a file that is specified as type ``OBS6'' in the Name File. For extracting model values associated with a package, input is read from a file designated by the keyword ``OBS6'' in the Options block of the package of interest. The structures of observation input files for models and packages do not differ. Where a file name (or path name) containing spaces is to be read, enclose the name in single quotation marks.

Each OBS6 file can contain an OPTIONS block and one or more CONTINUOUS blocks. Each OBS6 file must contain at least one block. If present, the OPTIONS block must appear first. The CONTINUOUS blocks can be listed in any order. Comments, indicated by the presence of the ``\#'' character in column 1, can appear anywhere in the file and are ignored. 

Observations are output at the end of each time step and represent the value used by \mf during the time step. When input to the OBS utility references a stress-package boundary (for packages other than the advanced stress packages) that is not defined for a stress period of interest, a special NODATA value, indicating that a simulated value is not available, is written to output. The NODATA value is $3.0 \times 10\textsuperscript{30}$. 

Output files to be generated by the Observation utility can be either text or binary. When a text file is used for output, the user can specify the number of digits of precision are to be used in writing values. For compatibility with common spreadsheet programs, text files are written in Comma-Separated Values (CSV) format. For this reason, text output files are commonly named with ``csv'' as the extension. By convention, binary output files are named with ``bsv'' (for ``binary simulated values'') as the extension.

%When a binary file is used, the user can specify whether floating-point numbers should be written in single or double precision.

%For CONTINUOUS observations, note that boundaries identified by ID (and ID2 where used) must be defined in the corresponding package input file in all stress periods of the simulation. This requirement may mean that in some PERIOD blocks, the user will need to include entries that have no affect on the model; for example one could include a well with a recharge rate of zero or a drain boundary with a conductance of zero. In some situations preparation of input can be simplified by splitting package input into multiple input files, so that boundaries included in CONTINUOUS observations are separated from other boundaries simulated by the same package type.

\subsection{Structure of Blocks}
\vspace{5mm}

\noindent \textit{FOR EACH SIMULATION}
\lstinputlisting[style=blockdefinition]{./mf6ivar/tex/utl-obs-options.dat}
\lstinputlisting[style=blockdefinition]{./mf6ivar/tex/utl-obs-continuous.dat}

\subsection{Explanation of Variables}
\begin{description}
% DO NOT MODIFY THIS FILE DIRECTLY.  IT IS CREATED BY mf6ivar.py 

\item \textbf{Block: OPTIONS}

\begin{description}
\item \texttt{digits}---Keyword and an integer digits specifier used for conversion of simulated values to text on output. If not specified, the default is the maximum number of digits stored in the program (as written with the G0 Fortran specifier). When simulated values are written to a comma-separated value text file specified in a CONTINUOUS block below, the digits specifier controls the number of significant digits with which simulated values are written to the output file. The digits specifier has no effect on the number of significant digits with which the simulation time is written for continuous observations.  If DIGITS is specified as zero, then observations are written with the default setting, which is the maximum number of digits.

\item \texttt{PRINT\_INPUT}---keyword to indicate that the list of observation information will be written to the listing file immediately after it is read.

\end{description}
\item \textbf{Block: CONTINUOUS}

\begin{description}
\item \texttt{FILEOUT}---keyword to specify that an output filename is expected next.

\item \texttt{obs\_output\_file\_name}---Name of a file to which simulated values corresponding to observations in the block are to be written. The file name can be an absolute or relative path name. A unique output file must be specified for each CONTINUOUS block. If the ``BINARY'' option is used, output is written in binary form. By convention, text output files have the extension ``csv'' (for ``Comma-Separated Values'') and binary output files have the extension ``bsv'' (for ``Binary Simulated Values'').

\item \texttt{BINARY}---an optional keyword used to indicate that the output file should be written in binary (unformatted) form.

\item \texttt{obsname}---string of 1 to 40 nonblank characters used to identify the observation. The identifier need not be unique; however, identification and post-processing of observations in the output files are facilitated if each observation is given a unique name.

\item \texttt{obstype}---a string of characters used to identify the observation type.

\item \texttt{id}---Text identifying cell where observation is located. For packages other than NPF, if boundary names are defined in the corresponding package input file, ID can be a boundary name. Otherwise ID is a cellid. If the model discretization is type DIS, cellid is three integers (layer, row, column). If the discretization is DISV, cellid is two integers (layer, cell number). If the discretization is DISU, cellid is one integer (node number).

\item \texttt{id2}---Text identifying cell adjacent to cell identified by ID. The form of ID2 is as described for ID. ID2 is used for intercell-flow observations of a GWF model, for three observation types of the LAK Package, for two observation types of the MAW Package, and one observation type of the UZF Package.

\end{description}


\end{description}


\subsection{Available Observation Types}

\subsubsection{GWF Observations}
\input{./obs/obs-gwf.tex}

\subsubsection{GWT Observations}
\input{./obs/obs-gwt.tex}

\subsubsection{GWE Observations}
Observations are available for GWE models and GWE stress packages. Available observation types have been listed for each package that supports observations (tables~\ref{table:gweobstype} to~\ref{table:gwe-cntobstype}). All available observation types are repeated in Table~\ref{table:gwe-obstypetable} for convenience. 

The sign convention adopted for transport observations are identical to the conventions used in budgets contained in listing files and used in the cell-by-cell budget output. For flow-ja-face observation types, negative and positive values represent a loss from and gain to the cellid specified for ID, respectively. For standard stress packages, negative and positive values represent a loss from and gain to the GWE model, respectively. For advanced transport packages (Package = LKE, MWE, SFE, and UZE), negative and positive values for exchanges with the GWE model (Observation type = lke, mwe, sfe, and uze) represent a loss from and gain to the GWE model, respectively. For other advanced stress package flow terms, negative and positive values represent a loss from and gain from the advanced package, respectively.

\FloatBarrier

\begingroup
\makeatletter
\ifx\LT@ii\@undefined\else
\def\LT@entry#1#2{\noexpand\LT@entry{-#1}{#2}}
\xdef\LT@i{\LT@ii}
\fi
\endgroup

% model observations
\begin{longtable}{p{2cm} p{2.75cm} p{2cm} p{1.25cm} p{7cm}}
\caption{Available observation types for the GWE Model} \tabularnewline

\hline
\hline
\textbf{Model} & \textbf{Observation types} & \textbf{ID} & \textbf{ID2} & \textbf{Description} \\
\hline
\endfirsthead

\captionsetup{textformat=simple}
\caption*{\textbf{Table \arabic{table}.}{\quad}List of symbols used in this report.---Continued} \\

\hline
\hline
\textbf{Model} & \textbf{Observation types} & \textbf{ID} & \textbf{ID2} & \textbf{Description} \\
\hline
\endhead

\hline
\endfoot

GWE & temperature & cellid & -- & Temperature at a specified cell. \\
GWE & flow-ja-face & cellid & cellid & Energy flow in dimensions of watts between two adjacent cells.  The energy flow rate includes the contributions from both advection and conduction (including mechanical dispersion) if those packages are active
\end{longtable}
\addtocounter{table}{-1}

% stress packages
\begin{longtable}{p{2cm} p{2.75cm} p{2cm} p{1.25cm} p{7cm}}
\hline
\hline
\textbf{Stress Package} & \textbf{Observation type} & \textbf{ID} & \textbf{ID2} & \textbf{Description} \\
\hline
\endfirsthead

\captionsetup{textformat=simple}
\caption*{\textbf{Table \arabic{table}.}{\quad}Available GWE observation types.---Continued} \\

\hline
\hline
\textbf{Stress Package} & \textbf{Observation types} & \textbf{ID} & \textbf{ID2} & \textbf{Description} \\
\hline
\endhead

\hline
\endfoot

CTP & ctp & cellid or boundname & -- & Energy flow between the groundwater system and a constant-temperature boundary or a group of cells with constant-temperature boundaries.
 \\
\hline
% ESR & esr & cellid or boundname & -- & Energy source loading rate between the groundwater system and a energy source loading boundary or a group of  boundaries. \\
% \hline
% general APT observations
SFE & temperature & rno or boundname & -- & Reach temperature. If boundname is specified, boundname must be unique for each reach. \\
SFE & flow-ja-face & rno or boundname & rno or -- & Energy flow between two reaches.  If a boundname is specified for ID1, then the result is the total energy flow for all reaches. If a boundname is specified for ID1 then ID2 is not used.\\
SFE & storage & rno or boundname & -- & Simulated energy storage flow rate for a reach or group of reaches. \\
SFE & constant & rno or boundname & -- & Simulated energy constant-flow rate for a reach or group of reaches. \\
SFE & from-mvr & rno or boundname & -- & Simulated energy inflow into a reach or group of reaches from the MVE package. Energy inflow is calculated as the product of provider temperature and the mover flow rate. \\
SFE & to-mvr & rno or boundname & -- & Energy outflow from a reach, or a group of reaches that is available for the MVR package. If boundname is not specified for ID, then the outflow available for the MVR package from a specific reach is observed. \\
SFE & sfe & rno or boundname & -- & Energy flow rate for a reach or group of reaches and its aquifer connection(s). \\

%observations specific to the stream energy transport package
% rainfall evaporation runoff ext-inflow withdrawal outflow
SFE & rainfall & rno or boundname & -- & Rainfall rate applied to a reach or group of reaches multiplied by the rainfall temperature. \\
SFE & evaporation & rno or boundname & -- & Simulated evaporation rate from a reach or group of reaches multiplied by the latent heat of vaporization for determining the amount of energy lost from a reach. \\
SFE & runoff & rno or boundname & -- & Runoff rate applied to a reach or group of reaches multiplied by the runoff temperature. \\
SFE & ext-inflow & rno or boundname & -- & Energy inflow into a reach or group of reaches calculated as the external inflow rate multiplied by the inflow temperature. \\
SFE & ext-outflow & rno or boundname & -- & External outflow from a reach or group of reaches to an external boundary. If boundname is not specified for ID, then the external outflow from a specific reach is observed. In this case, ID is the reach rno. \\
SFE & strmbd-cond & rno or boundname & -- & Amount of heat conductively exchanged with the streambed material. 
 \\
\hline
% % general APT observations
LKE & temperature & lakeno or boundname & -- & Lake temperature. If boundname is specified, boundname must be unique for each lake. \\
LKE & flow-ja-face & lakeno or boundname & lakeno or -- & Energy flow between two lakes connected by an outlet.  If more than one outlet is used to connect the same two lakes, then the energy flow for only the first outlet can be observed.  If a boundname is specified for ID1, then the result is the total energy flow for all outlets for a lake. If a boundname is specified for ID1 then ID2 is not used.\\
LKE & storage & lakeno or boundname & -- & Simulated energy storage flow rate for a lake or group of lakes. \\
LKE & constant & lakeno or boundname & -- & Simulated energy constant-flow rate for a lake or group of lakes. \\
LKE & from-mvr & lakeno or boundname & -- & Simulated energy inflow into a lake or group of lakes from the MVE package. Energy inflow is calculated as the product of provider temperature and the mover flow rate. \\
LKE & to-mvr & outletno or boundname & -- & Energy outflow from a lake outlet, a lake, or a group of lakes that is available for the MVR package. If boundname is not specified for ID, then the outflow available for the MVR package from a specific lake outlet is observed. In this case, ID is the outlet number, which must be between 1 and NOUTLETS. \\
LKE & lke & lakeno or boundname & \texttt{iconn} or -- & Energy flow rate for a lake or group of lakes and its aquifer connection(s). If boundname is not specified for ID, then the simulated lake-aquifer flow rate at a specific lake connection is observed. In this case, ID2 must be specified and is the connection number \texttt{iconn} for lake \texttt{lakeno}. \\

%observations specific to the lake package
% rainfall evaporation runoff ext-inflow withdrawal outflow
LKE & rainfall & lakeno or boundname & -- & Rainfall rate applied to a lake or group of lakes multiplied by the rainfall temperature. \\
LKE & evaporation & lakeno or boundname & -- & Simulated evaporation rate from a lake or group of lakes multiplied by the latent heat of evaporation for determining the energy lost from a lake. \\
LKE & runoff & lakeno or boundname & -- & Runoff rate applied to a lake or group of lakes multiplied by the runoff temperature. \\
LKE & ext-inflow & lakeno or boundname & -- & Energy inflow into a lake or group of lakes calculated as the external inflow rate multiplied by the inflow temperature. \\
LKE & withdrawal & lakeno or boundname & -- & Specified withdrawal rate from a lake or group of lakes multiplied by the simulated lake temperature. \\
LKE & ext-outflow & lakeno or boundname & -- & External outflow from a lake or a group of lakes, through their outlets, to an external boundary.  If the water mover is active, the reported ext-outflow value plus the rate to mover is equal to the total outlet outflow.

 \\
% \hline
% % general APT observations
MWE & temperature & mawno or boundname & -- & Well temperature. If boundname is specified, boundname must be unique for each well. \\
%flowjaface not included
MWE & storage & mawno or boundname & -- & Simulated energy storage flow rate for a well or group of wells. \\
MWE & constant & mawno or boundname & -- & Simulated energy constant-flow rate for a well or group of wells. \\
MWE & from-mvr & mawno or boundname & -- & Simulated energy inflow into a well or group of wells from the MVE package. Energy inflow is calculated as the product of provider temperature and the mover flow rate. \\
MWE & mwe & mawno or boundname & \texttt{iconn} or -- & Energy flow rate for a well or group of wells and its aquifer connection(s). If boundname is not specified for ID, then the simulated well-aquifer flow rate at a specific well connection is observed. In this case, ID2 must be specified and is the connection number \texttt{iconn} for well \texttt{mawno}. \\

% observations specific to the mwe package
MWE & rate & mawno or boundname & -- & Simulated energy flow rate for a well or group of wells. \\
MWE & fw-rate & mawno or boundname & -- & Simulated energy flow rate for a flowing well or group of flowing wells. \\
MWE & rate-to-mvr & well or boundname & -- & Simulated energy flow rate that is sent to the MVE Package for a well or group of wells.\\
MWE & fw-rate-to-mvr & well or boundname & -- & Simulated energy flow rate that is sent to the MVE Package from a flowing well or group of flowing wells. \\
 \\
\hline
% general APT observations
UZE & temperature & uzeno or boundname & -- & uze cell temperature. If boundname is specified, boundname must be unique for each uze cell. \\
UZE & flow-ja-face & uzeno or boundname & uzeno or -- & Energy flow between two uze cells.  If a boundname is specified for ID1, then the result is the total energy flow for all uze cells. If a boundname is specified for ID1 then ID2 is not used.\\
UZE & storage & uzeno or boundname & -- & Simulated energy storage flow rate for a uze cell or group of uze cells. \\
UZE & constant & uzeno or boundname & -- & Simulated energy constant-flow rate for a uze cell or a group of uze cells. \\
UZE & from-mvr & uzeno or boundname & -- & Simulated energy inflow into a uze cell or group of uze cells from the MVE package. Energy inflow is calculated as the product of provider temperature and the mover flow rate. \\
UZE & uze & uzeno or boundname & -- & Energy flow rate for a uze cell or group of uze cells and its aquifer connection(s). \\

%observations specific to the uze package
% infiltration rej-inf uzet rej-inf-to-mvr
UZE & infiltration & uzeno or boundname & -- & Infiltration rate applied to a uze cell or group of uze cells multiplied by the infiltration temperature. \\
UZE & rej-inf & uzeno or boundname & -- & Rejected infiltration rate applied to a uze cell or group of uze cells multiplied by the infiltration temperature. \\
UZE & uzet & uzeno or boundname & -- & Unsaturated zone evapotranspiration rate applied to a uze cell or group of uze cells multiplied by the uze cell temperature. \\
UZE & rej-inf-to-mvr & uzeno or boundname & -- & Rejected infiltration rate applied to a uze cell or group of uze cells multiplied by the infiltration temperature that is sent to the mover package. \\

\label{table:gwe-obstypetable}
\end{longtable}
\addtocounter{table}{-1}

% exchange
\begin{longtable}{p{2cm} p{2.75cm} p{2cm} p{1.25cm} p{7cm}}
\hline
\hline
\textbf{Exchange} & \textbf{Observation type} & \textbf{ID} & \textbf{ID2} & \textbf{Description} \\
\hline
\endfirsthead

\captionsetup{textformat=simple}
\caption*{\textbf{Table \arabic{table}.}{\quad}Available GWE observation types.---Continued} \\

\hline
\hline
\textbf{Exchange} & \textbf{Observation types} & \textbf{ID} & \textbf{ID2} & \textbf{Description} \\
\hline
\endhead

\hline
\endfoot

GWE-GWE & flow-ja-face & exchange number or boundname & -- & Energy flow between model 1 and model 2 for a specified exchange (which is the consecutive exchange number listed in the EXCHANGEDATA block), or the sum of these exchange flows by boundname if boundname is specified.
\end{longtable}

\normalsize

\FloatBarrier

