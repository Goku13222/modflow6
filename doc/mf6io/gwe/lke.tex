Lake Energy Transport (LKE) Package information is read from the file that is specified by ``LKE6'' as the file type.  There can be as many LKE Packages as necessary for a GWE model. Each LKE Package is designed to work with flows from a single corresponding GWF LAK Package. By default \mf uses the LKE package name to determine which LAK Package corresponds to the LKE Package.  Therefore, the package name of the LKE Package (as specified in the GWE name file) must match with the name of the corresponding LAK Package (as specified in the GWF name file).  Alternatively, the name of the flow package can be specified using the FLOW\_PACKAGE\_NAME keyword in the options block.  The GWE LKE Package cannot be used without a corresponding GWF LAK Package.

The LKE Package does not have a dimensions block; instead, dimensions for the LKE Package are set using the dimensions from the corresponding LAK Package.  For example, the LAK Package requires specification of the number of lakes (NLAKES).  LKE sets the number of lakes equal to NLAKES.  Therefore, the PACKAGEDATA block below must have NLAKES entries in it.

\vspace{5mm}
\subsubsection{Structure of Blocks}
\lstinputlisting[style=blockdefinition]{./mf6ivar/tex/gwe-lke-options.dat}
\lstinputlisting[style=blockdefinition]{./mf6ivar/tex/gwe-lke-packagedata.dat}
\lstinputlisting[style=blockdefinition]{./mf6ivar/tex/gwe-lke-period.dat}

\vspace{5mm}
\subsubsection{Explanation of Variables}
\begin{description}
% DO NOT MODIFY THIS FILE DIRECTLY.  IT IS CREATED BY mf6ivar.py 

\item \textbf{Block: OPTIONS}

\begin{description}
\item \texttt{flow\_package\_name}---keyword to specify the name of the corresponding flow package.  If not specified, then the corresponding flow package must have the same name as this advanced transport package (the name associated with this package in the GWE name file).

\item \texttt{auxiliary}---defines an array of one or more auxiliary variable names.  There is no limit on the number of auxiliary variables that can be provided on this line; however, lists of information provided in subsequent blocks must have a column of data for each auxiliary variable name defined here.   The number of auxiliary variables detected on this line determines the value for naux.  Comments cannot be provided anywhere on this line as they will be interpreted as auxiliary variable names.  Auxiliary variables may not be used by the package, but they will be available for use by other parts of the program.  The program will terminate with an error if auxiliary variables are specified on more than one line in the options block.

\item \texttt{flow\_package\_auxiliary\_name}---keyword to specify the name of an auxiliary variable in the corresponding flow package.  If specified, then the simulated temperatures from this advanced transport package will be copied into the auxiliary variable specified with this name.  Note that the flow package must have an auxiliary variable with this name or the program will terminate with an error.  If the flows for this advanced transport package are read from a file, then this option will have no effect.

\item \texttt{BOUNDNAMES}---keyword to indicate that boundary names may be provided with the list of lake cells.

\item \texttt{PRINT\_INPUT}---keyword to indicate that the list of lake information will be written to the listing file immediately after it is read.

\item \texttt{PRINT\_TEMPERATURE}---keyword to indicate that the list of lake temperature will be printed to the listing file for every stress period in which ``TEMPERATURE PRINT'' is specified in Output Control.  If there is no Output Control option and PRINT\_TEMPERATURE is specified, then temperature are printed for the last time step of each stress period.

\item \texttt{PRINT\_FLOWS}---keyword to indicate that the list of lake flow rates will be printed to the listing file for every stress period time step in which ``BUDGET PRINT'' is specified in Output Control.  If there is no Output Control option and ``PRINT\_FLOWS'' is specified, then flow rates are printed for the last time step of each stress period.

\item \texttt{SAVE\_FLOWS}---keyword to indicate that lake flow terms will be written to the file specified with ``BUDGET FILEOUT'' in Output Control.

\item \texttt{TEMPERATURE}---keyword to specify that record corresponds to temperature.

\item \texttt{tempfile}---name of the binary output file to write temperature information.

\item \texttt{BUDGET}---keyword to specify that record corresponds to the budget.

\item \texttt{FILEOUT}---keyword to specify that an output filename is expected next.

\item \texttt{budgetfile}---name of the binary output file to write budget information.

\item \texttt{BUDGETCSV}---keyword to specify that record corresponds to the budget CSV.

\item \texttt{budgetcsvfile}---name of the comma-separated value (CSV) output file to write budget summary information.  A budget summary record will be written to this file for each time step of the simulation.

\item \texttt{TS6}---keyword to specify that record corresponds to a time-series file.

\item \texttt{FILEIN}---keyword to specify that an input filename is expected next.

\item \texttt{ts6\_filename}---defines a time-series file defining time series that can be used to assign time-varying values. See the ``Time-Variable Input'' section for instructions on using the time-series capability.

\item \texttt{OBS6}---keyword to specify that record corresponds to an observations file.

\item \texttt{obs6\_filename}---name of input file to define observations for the LKE package. See the ``Observation utility'' section for instructions for preparing observation input files. Tables \ref{table:gwf-obstypetable} and \ref{table:gwt-obstypetable} lists observation type(s) supported by the LKE package.

\end{description}
\item \textbf{Block: PACKAGEDATA}

\begin{description}
\item \texttt{lakeno}---integer value that defines the lake number associated with the specified PACKAGEDATA data on the line. LAKENO must be greater than zero and less than or equal to NLAKES. Lake information must be specified for every lake or the program will terminate with an error.  The program will also terminate with an error if information for a lake is specified more than once.

\item \texttt{strt}---real value that defines the starting temperature for the lake.

\item \texttt{ktf}---is the thermal conductivity of the of the interface between the aquifer cell and the lake.

\item \texttt{rbthcnd}---real value that defines the thickness of the lakebed material through which conduction occurs.  Must be greater than 0.

\item \textcolor{blue}{\texttt{aux}---represents the values of the auxiliary variables for each lake. The values of auxiliary variables must be present for each lake. The values must be specified in the order of the auxiliary variables specified in the OPTIONS block.  If the package supports time series and the Options block includes a TIMESERIESFILE entry (see the ``Time-Variable Input'' section), values can be obtained from a time series by entering the time-series name in place of a numeric value.}

\item \texttt{boundname}---name of the lake cell.  BOUNDNAME is an ASCII character variable that can contain as many as 40 characters.  If BOUNDNAME contains spaces in it, then the entire name must be enclosed within single quotes.

\end{description}
\item \textbf{Block: PERIOD}

\begin{description}
\item \texttt{iper}---integer value specifying the starting stress period number for which the data specified in the PERIOD block apply.  IPER must be less than or equal to NPER in the TDIS Package and greater than zero.  The IPER value assigned to a stress period block must be greater than the IPER value assigned for the previous PERIOD block.  The information specified in the PERIOD block will continue to apply for all subsequent stress periods, unless the program encounters another PERIOD block.

\item \texttt{lakeno}---integer value that defines the lake number associated with the specified PERIOD data on the line. LAKENO must be greater than zero and less than or equal to NLAKES.

\item \texttt{laksetting}---line of information that is parsed into a keyword and values.  Keyword values that can be used to start the LAKSETTING string include: STATUS, TEMPERATURE, RAINFALL, EVAPORATION, RUNOFF, and AUXILIARY.  These settings are used to assign the temperature associated with the corresponding flow terms.  Temperatures cannot be specified for all flow terms.  For example, the Lake Package supports a ``WITHDRAWAL'' flow term.  If this withdrawal term is active, then water will be withdrawn from the lake at the calculated temperature of the lake.

\begin{lstlisting}[style=blockdefinition]
STATUS <status>
TEMPERATURE <@temperature@>
RAINFALL <@rainfall@>
EVAPORATION <@evaporation@>
RUNOFF <@runoff@>
EXT-INFLOW <@ext-inflow@>
AUXILIARY <auxname> <@auxval@> 
\end{lstlisting}

\item \texttt{status}---keyword option to define lake status.  STATUS can be ACTIVE, INACTIVE, or CONSTANT. By default, STATUS is ACTIVE, which means that temperature will be calculated for the lake.  If a lake is inactive, then there will be no solute mass fluxes into or out of the lake and the inactive value will be written for the lake temperature.  If a lake is constant, then the temperature for the lake will be fixed at the user specified value.

\item \textcolor{blue}{\texttt{temperature}---real or character value that defines the temperature for the lake. The specified TEMPERATURE is only applied if the lake is a constant temperature lake. If the Options block includes a TIMESERIESFILE entry (see the ``Time-Variable Input'' section), values can be obtained from a time series by entering the time-series name in place of a numeric value.}

\item \textcolor{blue}{\texttt{rainfall}---real or character value that defines the rainfall temperature for the lake. If the Options block includes a TIMESERIESFILE entry (see the ``Time-Variable Input'' section), values can be obtained from a time series by entering the time-series name in place of a numeric value.}

\item \textcolor{blue}{\texttt{evaporation}---real or character value that defines the temperature of evaporated water $(^{\circ}C)$ for the reach. If this temperature value is larger than the simulated temperature in the reach, then the evaporated water will be removed at the same temperature as the reach.  If the Options block includes a TIMESERIESFILE entry (see the ``Time-Variable Input'' section), values can be obtained from a time series by entering the time-series name in place of a numeric value.}

\item \textcolor{blue}{\texttt{runoff}---real or character value that defines the temperature of runoff for the lake. Value must be greater than or equal to zero. If the Options block includes a TIMESERIESFILE entry (see the ``Time-Variable Input'' section), values can be obtained from a time series by entering the time-series name in place of a numeric value.}

\item \textcolor{blue}{\texttt{ext-inflow}---real or character value that defines the temperature of external inflow for the lake. Value must be greater than or equal to zero. If the Options block includes a TIMESERIESFILE entry (see the ``Time-Variable Input'' section), values can be obtained from a time series by entering the time-series name in place of a numeric value.}

\item \texttt{AUXILIARY}---keyword for specifying auxiliary variable.

\item \texttt{auxname}---name for the auxiliary variable to be assigned AUXVAL.  AUXNAME must match one of the auxiliary variable names defined in the OPTIONS block. If AUXNAME does not match one of the auxiliary variable names defined in the OPTIONS block the data are ignored.

\item \textcolor{blue}{\texttt{auxval}---value for the auxiliary variable. If the Options block includes a TIMESERIESFILE entry (see the ``Time-Variable Input'' section), values can be obtained from a time series by entering the time-series name in place of a numeric value.}

\end{description}


\end{description}

\vspace{5mm}
\subsubsection{Example Input File}
\lstinputlisting[style=inputfile]{./mf6ivar/examples/gwe-lke-example.dat}

\vspace{5mm}
\subsubsection{Available observation types}
Lake Energy Transport Package observations include lake temperature and all of the terms that contribute to the continuity equation for each lake. Additional LKE Package observations include energy flow rates for individual outlets, lakes, or groups of lakes (\texttt{outlet}). The data required for each LKE Package observation type is defined in table~\ref{table:gwe-lkeobstype}. Negative and positive values for \texttt{lke} observations represent a loss from and gain to the GWE model, respectively. For all other flow terms, negative and positive values represent a loss from and gain from the LKE package, respectively.

\begin{longtable}{p{2cm} p{2.75cm} p{2cm} p{1.25cm} p{7cm}}
\caption{Available LKE Package observation types} \tabularnewline

\hline
\hline
\textbf{Stress Package} & \textbf{Observation type} & \textbf{ID} & \textbf{ID2} & \textbf{Description} \\
\hline
\endfirsthead

\captionsetup{textformat=simple}
\caption*{\textbf{Table \arabic{table}.}{\quad}Available LKE Package observation types.---Continued} \tabularnewline

\hline
\hline
\textbf{Stress Package} & \textbf{Observation type} & \textbf{ID} & \textbf{ID2} & \textbf{Description} \\
\hline
\endhead


\hline
\endfoot

% general APT observations
LKE & temperature & lakeno or boundname & -- & Lake temperature. If boundname is specified, boundname must be unique for each lake. \\
LKE & flow-ja-face & lakeno or boundname & lakeno or -- & Energy flow between two lakes connected by an outlet.  If more than one outlet is used to connect the same two lakes, then the energy flow for only the first outlet can be observed.  If a boundname is specified for ID1, then the result is the total energy flow for all outlets for a lake. If a boundname is specified for ID1 then ID2 is not used.\\
LKE & storage & lakeno or boundname & -- & Simulated energy storage flow rate for a lake or group of lakes. \\
LKE & constant & lakeno or boundname & -- & Simulated energy constant-flow rate for a lake or group of lakes. \\
LKE & from-mvr & lakeno or boundname & -- & Simulated energy inflow into a lake or group of lakes from the MVE package. Energy inflow is calculated as the product of provider temperature and the mover flow rate. \\
LKE & to-mvr & outletno or boundname & -- & Energy outflow from a lake outlet, a lake, or a group of lakes that is available for the MVR package. If boundname is not specified for ID, then the outflow available for the MVR package from a specific lake outlet is observed. In this case, ID is the outlet number, which must be between 1 and NOUTLETS. \\
LKE & lke & lakeno or boundname & \texttt{iconn} or -- & Energy flow rate for a lake or group of lakes and its aquifer connection(s). If boundname is not specified for ID, then the simulated lake-aquifer flow rate at a specific lake connection is observed. In this case, ID2 must be specified and is the connection number \texttt{iconn} for lake \texttt{lakeno}. \\

%observations specific to the lake package
% rainfall evaporation runoff ext-inflow withdrawal outflow
LKE & rainfall & lakeno or boundname & -- & Rainfall rate applied to a lake or group of lakes multiplied by the rainfall temperature. \\
LKE & evaporation & lakeno or boundname & -- & Simulated evaporation rate from a lake or group of lakes multiplied by the latent heat of evaporation for determining the energy lost from a lake. \\
LKE & runoff & lakeno or boundname & -- & Runoff rate applied to a lake or group of lakes multiplied by the runoff temperature. \\
LKE & ext-inflow & lakeno or boundname & -- & Energy inflow into a lake or group of lakes calculated as the external inflow rate multiplied by the inflow temperature. \\
LKE & withdrawal & lakeno or boundname & -- & Specified withdrawal rate from a lake or group of lakes multiplied by the simulated lake temperature. \\
LKE & ext-outflow & lakeno or boundname & -- & External outflow from a lake or a group of lakes, through their outlets, to an external boundary.  If the water mover is active, the reported ext-outflow value plus the rate to mover is equal to the total outlet outflow.


\label{table:gwe-lkeobstype}
\end{longtable}

\vspace{5mm}
\subsubsection{Example Observation Input File}
\lstinputlisting[style=inputfile]{./mf6ivar/examples/gwe-lke-example-obs.dat}


